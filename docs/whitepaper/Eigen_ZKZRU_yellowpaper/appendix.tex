\appendix

%\section{Borromean ring signatures on Pedersen Commitment}
%\label{appendix:borr}
%Borromean ring signatures\cite{maxwell2015borromean} 
%https://blockstream.com/bitcoin17-final41.pdf

\section{Bulletproof protocol}
\label{appendix:bp}

BulletProof doesn't depends on trust setup. BulletProof use NUMS \ref{appendix:NUMS} strategy, where a hash function is utilized to compute the generator for Pedersen Commitment. The main building block of BulletProof is Inner Product Argument(IPA). We use curve Secp256k1 in our implementation.


\begin{algorithm}
 \DontPrintSemicolon
    \caption{Compute Generator: ComputesGenerators}
    \label{alg:ComputesGenerators}
    \LinesNumbered
    
    \KwIn{The elliptic curve public generator $g \in \GG$ and an integer n.}
    \KwOut{The set of generators for Pedersen Commitment (g, h, \textbf{g}, \textbf{h})}.
    Compute h = MapToGroup('something', p) \;
    i = 0 \;
    \While {i < n} {
        $c \sample ZZ_p$ \;
        $d \sample ZZ_p$ \;
        \textbf{g}[i] $\gets$ $c\cdot G$ \;
        \textbf{h}[i] $\gets$ $d\cdot G$ \;
        i $\gets$ i + 1\;
    }
    return (g, h, \textbf{g}, \textbf{h});
\end{algorithm}

\begin{algorithm}
 \DontPrintSemicolon
    \caption{$Setup_{IP}$}
    \label{alg:setup_ip}
    \LinesNumbered
    
    \KwIn{the set of generators(g, h, \textbf{g}, \textbf{h})}
    \KwOut{$params_{IP}$}
    
    u $\gets$  MapToGroup('something', p) \;
    return $params_{IP}$ = (g, h, \textbf{g}, \textbf{h}, u)
\end{algorithm}

\begin{algorithm}
 \DontPrintSemicolon
    \caption{$Setup_{RP}$}
    \label{alg:setup_rp}
    \LinesNumbered
    
    \KwIn{the input interval [a, b), and the field modulus p}.
    \KwOut{$params_{RP}$}
    \uIf{b is not a power of 2}{
        return "b must be a power of 2"\;
    } {
        $n \gets log_2b$ \;
        (g, h, \textbf{g}, \textbf{h}) $\gets$ ComputeGenerators(g, n) \;
        $params_{IP} \gets Setup_{IP}(g, h, \textbf{g}, \textbf{h})$ \;
        return $params_{RP}$ = ($params_{IP}$, n) \;
    }
\end{algorithm}

\begin{algorithm}
 \DontPrintSemicolon
    \caption{Vector Commitment: $Commit_{IP}$}
    \label{alg:vector_commitment}
    \LinesNumbered
    
    \KwIn{$params_{IP}$, \textbf{a, b}}.
    \KwOut{The commitment C}
    Compute C $\gets$ $g^ah^b \in \GG$ \;
    return C.
\end{algorithm}

\begin{algorithm}
 \DontPrintSemicolon
    \caption{Prove of Inner Product: $Prove_{IP}$}
    \label{alg:prove_ip}
    \LinesNumbered
    
    \KwIn{$params_{IP}, commit_{IP}$,c, \textbf{a, b}}.
    \KwOut{$proof_{IP}$}
    x $\gets$ Hash(\textbf{g, h}, P, c) $\in \GG$ \;
    Compute P' $\gets$ $u^{x\cdot c}P$ \;
    Allocate arrays \textbf{l, r} $\in \GG^n$ \;
    ComputeProof(\textbf{g}, \textbf{h}, P', $u^x$, \textbf{a, b, l, r})\;
    $proof_{IP}$ $\gets$ (\textbf{g, h}, P', $u^x$, \textbf{a, b, l, r}) \; 
    return $proof_{IP}$.
\end{algorithm}

\begin{algorithm}
    \caption{Prove of Inner Product: ComputeProof}
    \label{alg:compute_proof}
    \LinesNumbered
    \KwIn{\textbf{g, h}, P, u, \textbf{a, b, l, r}}
    \KwOut{\textbf{g, h}, P, u, \textbf{a, b, l, r}}
    x = Hash(g, h, P, c $\in \ZZ_p^*$\;
    Compute P' = $u^{x\cdot c}P$ \;
    \eIf{n = 1}{
        return (\textbf{g, h}, P, u, a, b, \textbf{l, r}) \;
    } {
        n' $\gets$ n/2 \;
        $C_L \gets <a_{[:n']}, b_{[n':]}> \in \ZZ_p$\;
        $C_R \gets <a_{[n':]}, b_{[:n']}> \in \ZZ_p$\;
        $L \gets g_{[n':]}^{a_{[:n']}} h_{[:n']}^{b_{[n':]}}u^{C_L} \in \GG$ \;
        $R \gets g_{[n':]}^{a_{[n':]}} h_{[n':]}^{b_{[:n']}}u^{C_R} \in \GG$ \;
        Append $L, R$ to \textbf{l, r}, respectively\;
        x $\gets$ Hash($L, R$)\;
        $g' \gets g_{[:n']}^{x^{-1}} g_{[n':]}^{x} \in \GG^{n'}$ \;
        $h' \gets g_{[:n']}^{x} g_{[n':]}^{x^{-1}} \in \GG^{n'}$ \;
        $P' \gets L^{x^2}PR^{x^{-2}} \in \GG$ \;
        $a' \gets a_{[:n']}x + a_{[n':]}x^{-1} \in \ZZ_p^{n'}$ \;
        $b' \gets b_{[:n']}x + b_{[n':]}x^{-1} \in \ZZ_p^{n'}$ \;
        return ComputeProof(\textbf{g', h'}, P', u, \textbf{a', b', l, r})
    }
\end{algorithm}

\begin{algorithm}
 \DontPrintSemicolon
    \caption{Prove of Inner Product: $Verify_{IP}$}
    \label{alg:verify_ip}
    \LinesNumbered
    
    \KwIn{$params_{IP}, commit_{IP}, proof_{IP}$}
    \KwOut{True or false}
    
    i $\gets$ 0 \;
    \While{i < logn}{
        n' $\gets$ n/2 \;
        x $\gets$ Hash(\textbf{l}[i], \textbf{r}[i]) \;
        $g' \gets g_{[:n']}^{x^{-1}} g_{[n':]}^{x} \in \GG^{n'}$ \;
        $h' \gets g_{[:n']}^{x} g_{[n':]}^{x^{-1}} \in \GG^{n'}$ \;
        $P' \gets L^{x^2}P R^{x^{-2}}\in \GG$ \;
        i $\gets$ i + 1 \;
    }
    The verifier computes c = a$\cdot$ and accepts if $P = g^ah^bu^c$.
\end{algorithm}

% https://arxiv.org/pdf/1907.06381.pdf
\begin{algorithm}
 \DontPrintSemicolon
    \caption{Bulletproofs: $Prove_{RP}$}
    \label{alg:prove_rp}
    \LinesNumbered
    
    \KwIn{$params_{RP}$, v}
    \KwOut{$proof_{RP}$}
    
    $\lambda \sample \ZZ_p$ \;
    $V \gets g^vh^\lambda \in \GG$ \;
    $a_L \in \{\bin\}^n$ s.t. $<a_L, 2^n> = v$ \;
    $a_R - a_L - 1^n \in \ZZ+p^n$ \;
    $\alpha \sample \ZZ_p$\;
    $A \gets h^\alpha g^{a_L}h^{a_R} \in \GG$ \;
    $s_L, s_R \sample \ZZ_p^n$\;
    $\rho \sample \ZZ_p$\;
    $S \gets h^\rho g^{s_L} h^{s_R} \in \GG$ \;
    $ y \gets Hash(A, S) \in \ZZ_p^n$ \;
    $ z \gets Hash(A, S, y) \in \ZZ_p^n$ \;
    $\tau_1, \tau_2 \sample \ZZ_p$ \;
    $T_1 \gets g^{t_1} h^{\tau_1}$ \;
    $T_2 \gets g^{t_2} h^{\tau_2}$ \;
    $ x \gets Hash(T_1, T_2) \in \ZZ_p^n$ \;
    
    \textbf{l} $\gets$ $l(X) = a_L - z1^n + s_LX \in \ZZ_p^n$ \;
    \textbf{r} $\gets$ $r(x) = y^n \cdot (a_R + z1^n + s_R X) + z^22^n \in \ZZ_p^n$ \;
    $\hat{t} \gets $ <\textbf{l, r}> $\in \ZZ_p$ \;
    $\tau_x \gets r_2x^2 + \tau_1 x + z^2\lambda  \in \ZZ_p$ \;
    $\mu \gets \alpha + \rho x \in \ZZ_p$ \;
    $commit_{IP} \gets Commit_{IP}(params_{IP}, l, r)$ \;
    $proof_{IP} \gets Prove_{IP}(params_{IP}, commit_{IP}, \hat{t}, l, r)$ \;
    return $proof_{RP} = Prove_{RP}(\tau_x, \mu, \hat{t}, V, A, S, T_1, T_2, commit_{IP}, proof_{IP})$ \;
\end{algorithm}

\begin{algorithm}
 \DontPrintSemicolon
    \caption{Bulletproofs: $Verify_{RP}$}
    \label{alg:verify_rp}
    \LinesNumbered

    \KwIn{$params_{RP}, proof_{rp}$}
    \KwOut{True or false}
    
    $ y \gets Hash(A, S) \in \ZZ_p^n$ \;
    $ z \gets Hash(A, S, y) \in \ZZ_p^n$ \;
    $ x \gets Hash(T_1, T_2) \in \ZZ_p^n$ \;
    
    $h_i \gets h_i^{y^{-i + 1}} \in \GG, \forall i \in [1, n]$\;
    $P_l \gets p \cdot h^\mu $\;
    $P_r \gets A\cdot S^x\cdot g^{-z} \cdot (h')^{z\cdot y^2 + z^2 \cdot 2^n} \in \GG$\;
    $output_1 \gets (P_l \overset{?}{=}P_r)$ \;
    $output_2 \gets (g^{\hat{t}} h^{r_x} \overset{?}{=} V^{z^2}\cdot g^{\sigma(y, z)\cdot T_1^x \cdot T_2^{x^2}})$ \;
    $output_3 \gets Verfify_{IP}(proof_{IP})$ \;
    return $output_1 \land output_2 \land output_3$ 
\end{algorithm}

\section{Boneh-Boyen Signatures without pairings}
\label{appendix: bbs-no-pairing}
Boneh-Boyen Signatures scheme \cite{jao2009boneh} used by the designated authority
(which could be the verifier) to sign each element of the set $\Phi$ is the one proposed by Boneh and Boyen.
Based on this scheme, which is secure under the q-SDH assumption, it is possible to prove knowledge of a
signature on a message, without revealing the signature nor the message.

For a given secret x, and two random generatos $g_1, g_2$ of $G_1$, the signature of a message $m \in \ZZ_p$ is obtained by computing $\sigma = g_1^{1/(x+m)}$. Given a bilinear pairing e, a signature $\sigma$ of m is valid if $e(\sigma, yg_2^m) = e(g_1, g_2)$ holds, where $y = g_2^x$.

\cite{arfaoui2015practical} proposed BB signature without pairing under the DDH assumption. Let G be a cyclic group with prime order p where the
Decisional Diffie-Hellman (DDH) problem is assumed to be hard and $g_1, g_2$ two random generators of $\GG$. The signer’s private key is $x \in \ZZ_p$ and it's public key is $y = g^x$, so the signature would be $\sigma = g_1^{1/(x+m)}$, which implies that $A^x = g_1 A^{-m}$. So the signer can prove the signature is valid by generating a ZKPK $\pi$:
\begin{equation}
    SOK(x: X = g_2^x \land A^x = g_1A^{-m})
\end{equation}
which means the discrete logarithm of ($g_1 A^{-m}$) in the base A is equal to the discrete logarithm of $x$ in the base $g_2$.

\section{Set Membership Proof on BB Signature without pairing}

\begin{myDef}
\label{d5}
\textbf{Borreomean ring signature\cite{maxwell2015borromean}} let $\mathcal{V}$ be some set of verification keys, and f be a function which maps finite subsets of $\mathcal{V}$ to ${0, 1}$, we call f an admissibility function, then an admissible set V of verification keys is one for which f($\mathcal{V}$) = 1. 

A Borromean ring signature $\sigma$ is a signature on a message m with a set V of verification keys and admissibility function f which satisfy the following:
\begin{enumerate}
    \item $\sigma$ can be produced only by parties who together know all the secret keys to an admissible set $\mathcal{V}$ of verification keys.
    \item Given only $\sigma$ , $\mathcal{V}$ , and $m$, it is statistically indistinguishable which admissible set V was used.
\end{enumerate}
\end{myDef}

Proving knowledge of a valid BB Signature $A_k = g^{1/{y+k}}$ on a value $k$ commited in $Com = g_1^k h^v$, without using parings on the prover's side can be done in Algorithm \ref{alg:prove_sm} and Algorithm \ref{alg:verify_sm}. 

\begin{algorithm}
 \caption{Set Membership Proof: $Prove_{SM}$}
    \label{alg:prove_sm}
    \LinesNumbered
    
    \KwIn{challenge $ch$, public input=(set $\Phi$), private input = ($k, A_k = g^{1/{y+k}}$)}
    \KwOut{$proof_{SM}$}
    
    $v \sample \ZZ_p^*$\;
    $Com \gets g_1^k h^v$ \;
    $A_k \gets g^{1/(y+k)}$ \;
    $l \sample \ZZ_p^*$ \;
    $B \gets A_k^l, B_1 \gets B^{-1}, D \gets B_1^kg^l$ \;
    $k_1, l_1, r_1 \sample \ZZ_p^*$ \;
    $Com_1 \gets g_1^{k_1}h^{r_1}, D_1 = B_1^{k_1} g^{l_1}$ \;
    
    $C \gets H(Com, B, D, Com_1, D_1, ch)$ \;
    $s_1 \gets k_1 + c \times k \mod p$ \;
    $s_2 \gets r_1 + c \times r \mod p$ \;
    $s_3 \gets l_1 + c \times l \mod p$ \;
    
    $proof_{SM} = (Com, B, D, s_1, s_2, s_3)$
\end{algorithm}

\begin{algorithm}
 \caption{Set Membership Proof: $Verify_{SM}$}
    \label{alg:verify_sm}
    \LinesNumbered
    
    \KwIn{challenge $ch$, $proof_{SM}$}
    \KwOut{True or false}
    
    $\hat{C} \gets g_1^{s_1}h^{s_2}, \hat{D} = B_1^{s_1}g^{s_3}D^{-c}$ \;
    $output_1 \gets B \neq 1_{G_1}$ \;
    $output_2 \gets H(Com, B, D, \hat{C}, \hat{D}, ch)$ \;
    
    return $output_1 \land output_2$
\end{algorithm}

\section{Back-Maxwell Rangeproof}
\label{appendix:Back-Maxwell-Rangeproof}



\section{Nothing Up My Sleeve}
\label{appendix:NUMS}
This method enables the generation of verifiable random values. NUMS allows a prover to pick values in a way that demonstrates the values were not selected for a "nefarious purpose" - for example, to create a "backdoor" to the algorithm \cite{black2014elliptic}. The algorithm on curve Secp256k1 is described in Algorithm \ref{alg:nums}.

\begin{algorithm}
 \caption{Nothing Up My Sleeve: MapToGroup}
    \label{alg:nums}
    \LinesNumbered
    
    \KwIn{The input string m, and the filed prime modules p}
    \KwOut{AN elliptic curve point if successful or some error}
    i $\gets$ 0 \;
    \While {i < 256} {
        x $\gets$ Hash(m, i) \;
        rhs $\gets$ $x^3 + 7$ \;
        \uIf {rhs is a sqaure (mod p) } {
            $y \gets \sqrt{rhs}$ mod p \;
            \uIf{(x, y) is not the point at infinity}{
                return (x, y) \;
            }
        }
        i $\gets$ i + 1\;
    }
    return "Can not map to group"\;
\end{algorithm}