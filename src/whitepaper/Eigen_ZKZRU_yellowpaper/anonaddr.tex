
\subsection{Address Anonymity}

Eigen ZKZRU adopts fast dual key exchange based stealth address scheme for address anonymity. 

In general, the Stealth address and Nullifier are two widely used scheme for anonymous
transaction. Compared to Nullifier used by Tornado Cash and AZTEC, stealth address is account-friendly. Stealth address enables the sender to send the money to an "invisible" recipient, and the actual recipient can derive the private key of the "invisible" recipient's address. 

The widely-used Stealth Address protocol is Dual-Key StealthAddress Protocol (DKSAP), which is designed for a wallet solution \href{https://github.com/shadowproject}{ShadowSend} uses two pairs of keys, a scan keypair and a spend keypair to provide decentralized anonymous currency. However, the drawback of DKSAP is that it requires the receiver to continuously calculate and determine whether it is the real receiver of the transaction until it detected a transaction that matches. In this process, the receiver needs to perform many time-consuming elliptic curve scalar multiplication operations, which limits the application of DKSAP. Some new Enhanced DKSAP protocols \cite{courtois2017stealth} proposed more efficient approach to eliminate the intensive curve scalar multiplication, we optimizes the Enhanced DKSAP protocol by binding the \textit{txdata} to random r, and combining with the confidential transaction, we can directly move the assets of receivers in the transaction from the one-time address to real address in next section. 

In The Enhanced DKSAP protocol shown in Figure \ref{fig:edksap}, the recipient have $m$ keypairs ($b_i, B_i$) and one view keypair $(v, V)$ on group ($\GG, p, g$), m+1 hash function $H_i, i \in [0, m]$ for previous key, and none of those keys will be open. A proxy server/audit is necessary to monitor the transactions on chain, which includes some R value and computes the pk', and check whether pk' matches pk from corresponding transaction. For each public key $pk$, only the correct receiver can calculate the private key $H_0(S)+\sum{H_i(S)b_i}$ and spend the money. 

\begin{figure}[H]
\centering
\dbox{%
\procedure[linenumbering, colspace=-1cm]{The Enhanced DKSAP protocol}{%
\textbf{Sender} \>\>  \textbf{Auditor} \>\> \textbf{Receiver} \\
\>\>  \>\> (b_i, B_i)_{1 \leq i \leq m}, (v, V) \\
\>\>  \>\sendmessageleft*[1.5cm]{\text{\scriptsize View key v}}  \> \\
\>\>  V \gets vG  \>\>        \\
\>\sendmessageleft*[1cm]{\text{\scriptsize Scan pubkey V}}  \> \>\> \\
rr \sample \ZZ_p   \>\>   \>\> \\
r = H(txdata || rr) \>\>   \>\>    \\
R \gets rG, S \gets rV  \>\>     \>\>  \\
pk \gets H_0(S)G + \sum{H_i(S)}B_i  \>\>    \>\> \\
\> \sendmessageright*[1.5cm]{\text{\scriptsize TX(pk, R, ...)}}  \> \>\> \\
\>\>  pk' \gets H_0(v R)G + \sum{H_i(vR)} B_i \>\> \\
\>\>  pk' \overset{?}{=} pk  \>\>        \\
\>\>  \>\sendmessageright*[1cm]{\text{\scriptsize TX($pk'$,...)}}  \> \\
\>\>  \>\> sk = H_0(S)+\sum{H_i(S)b_i }}
}
\caption{The Enhanced DKSAP protocol}
\label{fig:edksap}
\end{figure}

We extend the language $L_{transfer}$ previously in definition \ref{l_transfer} to cut off the linkage between Stealth Address with it's original address by providing a zero knowledge proof.  

Normally, the withdrawal can be completed by proof of knowledge language \ref{sig_proof} on message M and public key PK.
\begin{equation}\label{sig_proof}
    \begin{aligned}
        L_{withdraw} = \left\{
                \begin{array}{c}
                        (x, w) = ((M, PK, to, amount), (SIG, k)) | \\ 
                    SIG = EdDSA\_Sign(k, M) \land PK = kG \land \\
                        EdDSA\_verify(PK, SIG, M) = true \land to.transfer(amount)
                \end{array}    
                    \right\} 
    \end{aligned}
\end{equation}

But the the PK in the public input can be still extracted and the rebuild the linkage to the receiver address $to$.
We simply define the Stealth Address's public key as $PK = (h + b)G$, where $h = H_0(rV), b = \sum{H_i(S)b_i}$. To shield the linkage between $PK$ and receiver address $to$, we must hide $PK$, so we can let the random part of $PK$ be the public input, and the $B=bG$ be the private input. Hence, the proof of knowledge language can be transformed to Eq. \ref{anon_withdraw}.

\begin{equation}\label{anon_withdraw}
    \begin{aligned}
        L_{anon\_withdraw} = \left\{
                \begin{array}{c}
                        (x, w) = ((M, H, to, amount), (SIG, h, b)) | \\ 
                    SIG = EdDSA\_Sign(h+b, M) \land H = hG \land B = bG \land \\
                        EdDSA\_verify(B + H, SIG, M) = true \land to.transfer(amount)
                \end{array}    
                    \right\} 
    \end{aligned}
\end{equation}

So anyone holding the public input $H$ and $to$ can not detect the linkage between $PK$ and $to$.

% \subsection{Combine Address Anonymity with Confidential Transaction} \label{section:caact}

% We extend the language $L_{transfer}$ previously in definition \ref{l_transfer} to support stealth address in definition \ref{l_transfer_ex}.  For simplicity, we always withdraw the asset in the stealth address to the $B_0$, whose balance is in encryption.

% \begin{equation}\label{l_transfer_ex}
% \begin{aligned}
%     L_{transfer} &= \{((pk_s, pk_r, C_s, C_r, sk_s, v, r_1, r_2, H_0, ..., H_m, R, v, B_1, ..., B_m)) | \\ 
%         pk_r &= H_0(v R)G + \sum{H_i(vR)} B_i \land \\
%         pk_s &= sk_sG \land C_s = \mbox{HPKE}.Enc(pk_s,v, r_1) \land \\
%         C_r &=  \mbox{HPKE}.Enc(pk_r,v, r_2) \land v \in [0, \mbox{MAX}) \land \\
%         &\mbox{HPKE}.Dec(sk_s, \hat{C}_s - C_s) \in [0, \mbox{MAX})\}
% \end{aligned}
% \end{equation}


% let l be a pre-consensus constant, and $L = 2^l - 1$, and generate the group parameters (\GG, p, g, h). In The Enhanced DKSAP protocol, the (s, S) is the spending key, and the (b, B) is the view key. 
% \procb[linenumbering]{The Enhanced DKSAP protocol}{
% \textbf{Sender} \> \> \textbf{Receiver} \\
% \> \> (s, S = g^s), c \sample \ZZ_p \\ 
% \> \> (b, c') = (H(v), H(c)), B = g^b \\
% \> \> i \sample [1, 2^{31}-1) \\
% \> \> S_i \gets f(S, c, i) \\
% \> \> B_i \gets f(B, c', i) \\
% \> \sendmessageleft*{\text{Stealth Address($S_i$, $B_i$, i})} \> \\ 
% rr \sample \ZZ_p \> \>    \\
% r = H(nonce_{sender} || txdata || rr) \> \>    \\
% R = g^r \> \>    \\
% R' = B_i^r\> \>    \\
% h = H(R')\> \>    \\
% P= S_i B_i h^{R' \cdot x}\> \>    \\
% i'= (((i >> l) \xor h) << l) + (i \& L)\> \>    \\
% Create\ TX(P, R, i',...)\> \>    \\
% \> \sendmessageright*{\text{TX(P, R, i', ...)}} \> \\ 
% \> \> b_i \gets f(b, c', i'\&L)   \\
% \> \> B_i = g^{b_i}, R' = R^{b_i}   \\
% \> \> h = H(R') \\
% \> \> i= (((i' >> l) \xor h) << l) + (i' \& L)    \\
% \> \> S_i = f(K, salt, i)  \\
% \> \> P'= S_i B_i h^{R' \cdot x}   \\
% \> \> Check\ P' \overset{?}{=} P 
% }
% Where $f$ is the Key Derivation Function, $H$ is hash function, and the input would be limited to size of $l$.
% We spilt the i into two parts. The lower part is of size $l$, and the higher part is used to hide the shared key' hash R.  The receiver publishes a stealth address $(S_i, B_i, i)$, and uses this address to generate a one-time public key for a transaction $TX$, attaches the (R, i')) to the $TX$, then the receiver collects it from the chain.

% The sender gets the receiver’s address info ($S_i, B_i, i$), generates a random secret number $r \in [2^{63} - 1]$ and calculate a Pedersen commitment $ C=  S_i B_i h^{R' \cdot x}, R' = B_i^r$, then the sender may use this commitment $C$ or $Hash(C)$ as the destination key for the output and packs the $(R, i)$ somewhere into the transaction. This improved stealth address scheme makes it possible to manage multiple stealth addresses in one wallet, therefore the user is able to share different addresses for different senders.

% \begin{algorithm}[t]
%     \caption{Ethereum account-friendly DKSAP protocol}
%     \label{alg:DKSAP}
%     \LinesNumbered
%     \KwIn{choose G as the base point of an elliptic curve group}

%     \begin{itemize}
%         \item The receiver:  two private/public key pairs (s, S) and (b, B), where S = s·G and B = b·G are ‘scan public key’ and ‘spend public key’;
%         \item The sender:  generates an ephemeral key pair (r, R), where r = H(message ++ nonce), and R = r·G and transmits it with the transaction. 
%     \end{itemize}
    
%     \KwOut {an anonymous recipient payment transaction}
    
    
%     Both sender and receiver Calculate shared secret using the ECDH:
%     $ c = H(r \cdot s \cdot G = H(r \cdot S) = H (s \cdot R)) $, where $H(\cdot)$ is an oracle hash function \;
    
%     The sender uses $c \cdot G + B$ as the ephemeral destination address for sending the payment\; 
    
%     The receiver checks whether some transaction has been sent to the ephemeral destination address above by checking whether the recipient address equals to $ c \cdot G + B = (c + b) \cdot G$. 
    
%     If there is a match, the payment can be spent using the corresponding private key c + b. Note that the ephemeral private key c + b can only be computed by the receiver. 
% \end{algorithm}

% In our DKSAP's scheme, we adopt a proxy server. The receiver can share the ‘scan private key’ s and the ‘spend public key’ B with the proxy server so that those entities can scan the blockchain transaction on behalf of the receiver. However, they are not able the compute the ephemeral private key c + b and spend the payment.